\documentclass[12pt, letterpaper]{article}

\title{Determining the power and efficiency\protect\\ of a solar cell}
\author{Timofei Ryko}
\date{08.09.21}

\usepackage{pgfplots}
\pgfplotsset{compat=1.18}
\usepackage{pgfplotstable}
\usepackage{circuitikz}
\usepackage{float}

\pgfplotstableset{
    every head row/.style={
        before row=\hline,after row=\hline},
    every last row/.style={
        after row=\hline},
    column type/.add={|}{},
    every last column/.style={column type/.add={}{|}}
}
\renewcommand{\arraystretch}{1.15}

\begin{document}

\maketitle
\section{Solar cell short circuit current dependence on illuminance.}
The solar cell was connected to a multimeter with two wires and the optimal direct current measuring range was chosen (2mA).
\par
Then, the lamp was mounted on a holder in the highest possible position and the solar cell was placed under it. The light source was lowered averagely 30 times and after every lowering the value of illuminance in lx and circuit current in mA were measured. During the experiment measured range of multimeter was changed to 20mA when it became too high.
\par
All the data were entered in Google Sheets, the table is shown on the page \pageref{tab:task1} (table \ref{tab:task1}). To find the value of power of light Pv in mW for each value of illuminance the equation $Pv (mW) = E \cdot S \cdot 3.5 \cdot 10^{-4}$ was used, where E is given in lx and the S in $cm^2$ ($S = 9cm^2$).
\par
To study the correlation between Pv and I a graph was drawn up and the equation of the trend line was calculated (figure \ref{fig:task1}).
% $R^2 = 0.998$
\par
The linear correlation between power of light and short circuit current is observed. This is due to the fact that more photons are sent, more electrons are excited by light travel through the solar cell -- light energy is converted into electrical energy.

\begin{table}
\centering
\pgfplotstabletypeset[
columns/E/.style={column name=E (lx)},
columns/I/.style={
    column name=I (mA),
    fixed zerofill,precision=3
    },
columns/Pv/.style={
    column name=Pv (mW),
    fixed zerofill,precision=3
    },
assign column name/.style={
    /pgfplots/table/column name={\textbf{#1}}
    }
]{task1.txt}
\caption{Data measured for task 1}
\label{tab:task1}
\end{table}

\begin{figure}
\centering
\begin{tikzpicture}[scale=1.3]
\begin{axis}[
    xlabel=Power of light (mW),
    ylabel=Short circuit current (mA),
    legend style={at={(0,-0.25)},anchor=north west,legend cell align=left}
]
\addplot+[
    only marks,
    mark=o,
    mark size=1pt
    ]
    table{task1.txt};
\addplot [thick, red] table[
    y={create col/linear regression={y=I}}
]
{task1.txt};

\addlegendentry{$I(Pv)$}
\addlegendentry{
$\pgfmathprintnumber{\pgfplotstableregressiona} \cdot I
\pgfmathprintnumber[print sign]{\pgfplotstableregressionb}$}

\end{axis}
\end{tikzpicture}
\caption{Short circuit current vs power of light}
\label{fig:task1}
\end{figure}

\newpage
\section{The current-voltage characteristic and the energy conversion efficiency of a solar cell}
To study the electrical properties of a solar cell the electrical circuit allowing to measure the current-voltage characteristic of the solar cell was built: a voltmeter and a resistor were connected in parallel with the solar cell and an ampermeeter.
\par
During the measurement voltage and current were different, but light intensity was constant. The lamp was brought on the height to give the illuminance 3390 lx. Electrical power was calculated
according to the Joule's law: $Pe (mW) = I(mA) \cdot U(V)$ and power of light was calculated using the same formula as in the task 1. The effectiveness was evaluated as follows: $\eta = \frac{Pe}{Pv} \cdot 100 \%$. All the results are presented in the table \ref{tab:task2} on the page \pageref{tab:task2}. The highest efficiency $7.728\%$ was observed at $R = 3000$.

\begin{table}[htb]
\centering
\pgfplotstabletypeset[
columns/R/.style={column name=R (Ohm)},
columns/U/.style={
    column name=U (V),
    fixed zerofill,precision=3
    },
columns/I/.style={
    column name=I (mA),
    fixed zerofill,precision=3
    },
columns/Pe/.style={
    column name=Pe (mW),
    fixed zerofill,precision=3
    },
columns/Eff/.style={
    column name=$\eta$ (\%),
    fixed zerofill,precision=3
    },
assign column name/.style={
    /pgfplots/table/column name={\textbf{#1}}
    }
]{task2.txt}
\caption{Electrical I-V characteristics of the single solar cell for light 3390 lx}
\label{tab:task2}
\end{table}
\newpage
\section{Connecting solar cells into a solar panel}
We studied two cases: solar cells were connected in series and in parallel. The solar cell was placed in the same place in both experiments, the light source was fixed to achieve constant illuminance equal to the illuminance in the previous task.
\begin{figure}[h]
\centering
\begin{circuitikz} \draw
    (0,0) to[battery] (0,4)
      -- (2,4) to[battery] (2,0)
      -- (0,0)
    (2,0) -- (6,0)
      to[european resistor] (6,4)
      to[rmeterwa, t=A] (2,4)
    (6,0) -- (10,0)
    to[rmeterwa, t=V] (10,4)
      -- (6,4)
    ;
\end{circuitikz}
\caption{Two solar cells connected to the circuit in parallel}
\label{fig:task3_parallel}
\end{figure}

\begin{figure}[h]
    \centering
    \begin{circuitikz} \draw
        (2,4) to[battery] (2,2)
          to[battery] (2,0)
        (2,0) -- (6,0)
          to[european resistor] (6,4)
          to[rmeterwa, t=A] (2,4)
        (6,0) -- (10,0)
        to[rmeterwa, t=V] (10,4)
          -- (6,4)
        ;
    \end{circuitikz}
    \caption{Two solar cells connected to the circuit in series}
    \label{fig:task3_series}
    \end{figure}

The illuminance was 3390 lx and $P_v$ was 10.58 mW, based on the formula that is given in the section 1. The obtained I-V characteristics are presented in the tables below.

\begin{table}[h]
    \centering
    \pgfplotstabletypeset[
    columns/R/.style={column name=R (Ohm)},
    columns/U/.style={
        column name=U (V),
        fixed zerofill,precision=3
        },
    columns/I/.style={
        column name=I (mA),
        fixed zerofill,precision=3
        },
    columns/Pe/.style={
        column name=Pe (mW),
        fixed zerofill,precision=3
        },
    columns/Eff/.style={
        column name=$\eta$ (\%),
        fixed zerofill,precision=3
        },
    assign column name/.style={
        /pgfplots/table/column name={\textbf{#1}}
        }
    ]{task3_parallel.txt}
    \caption{Electrical I-V characteristics of the two solar cells connected in parallel for light 3390 lx}
    \label{tab:task3_parallel}
\end{table}

\begin{table}[h]
    \centering
    \pgfplotstabletypeset[
        columns/R/.style={column name=R (Ohm)},
        columns/U/.style={
            column name=U (V),
            fixed zerofill,precision=3
            },
        columns/I/.style={
            column name=I (mA),
            fixed zerofill,precision=3
            },
        columns/Pe/.style={
            column name=Pe (mW),
            fixed zerofill,precision=3
            },
        columns/Eff/.style={
            column name=$\eta$ (\%),
            fixed zerofill,precision=3
            },
        assign column name/.style={
            /pgfplots/table/column name={\textbf{#1}}
            }
        ]{task3_series.txt}
    \caption{Electrical I-V characteristics of the two solar cells connected in series for light 3390 lx}
    \label{tab:task3_series}
\end{table}

Based on this data, the following curves were plotted: Current-Voltage , Power-Voltage, Efficiency-Voltage, Power-Resistance.

\clearpage
\centering
\begin{tikzpicture}
    \begin{axis}[
        title={I-U curves},
        xlabel={Voltage, V},
        ylabel={Current, mA},
        ymajorgrids=true,
        xmajorgrids=true,
        grid style=dashed,
        legend columns=3,
        legend style={at={(0.5,-0.35)},anchor=south}
        ]
    \legend{}
    \addplot[
        color=blue,
        mark=square,
        ]
        table[x = U, y = I]{task2.txt};
        \addlegendentry{single cell}
    
    \addplot[
        color=red,
        mark=triangle,
        ]
        table[x = U, y = I]{task3_series.txt};
        \addlegendentry{2 in series}
    
    \addplot[
        color=green,
        mark=*,
        ]
        table[x = U, y = I]{task3_parallel.txt};
        \addlegendentry{2 in parallel}
    \end{axis}
\end{tikzpicture}

\centering
\begin{tikzpicture}
    \begin{axis}[
        title={Pe-U curves},
        xlabel={Voltage, V},
        ylabel={Electrical power, mW},
        ymajorgrids=true,
        xmajorgrids=true,
        grid style=dashed,
        legend columns=3,
        legend style={at={(0.5,-0.35)},anchor=south}
        ]
    \legend{}
    \addplot[
        color=blue,
        mark=square,
        ]
        table[x = U, y = Pe]{task2.txt};
        \addlegendentry{single cell}
    
    \addplot[
        color=red,
        mark=triangle,
        ]
        table[x = U, y = Pe]{task3_series.txt};
        \addlegendentry{2 in series}
    
    \addplot[
        color=green,
        mark=*,
        ]
        table[x = U, y = Pe]{task3_parallel.txt};
        \addlegendentry{2 in parallel}
    \end{axis}
\end{tikzpicture}

\centering
\begin{tikzpicture}
    \begin{axis}[
        title={Efficiency \%},
        xlabel={Voltage, V},
        ylabel={Efficiency \%},
        ymajorgrids=true,
        xmajorgrids=true,
        grid style=dashed,
        legend columns=3,
        legend style={at={(0.5,-0.35)},anchor=south}
        ]
    \legend{}
    \addplot[
        color=blue,
        mark=square,
        ]
        table[x = U, y = Eff]{task2.txt};
        \addlegendentry{single cell}
    
    \addplot[
        color=red,
        mark=triangle,
        ]
        table[x = U, y = Eff]{task3_series.txt};
        \addlegendentry{2 in series}
    
    \addplot[
        color=green,
        mark=*,
        ]
        table[x = U, y = Eff]{task3_parallel.txt};
        \addlegendentry{2 in parallel}
    \end{axis}
\end{tikzpicture}

\centering
\begin{tikzpicture}
    \begin{axis}[
        title={Pe-R curves},
        xlabel={Resistance, $\Omega$},
        ylabel={Electrical power, mW},
        ymajorgrids=true,
        xmajorgrids=true,
        grid style=dashed,
        legend columns=3,
        legend style={at={(0.5,-0.35)},anchor=south}
        ]
    \legend{}
    \addplot[
        color=blue,
        mark=square,
        ]
        table[x = R, y = Pe]{task2.txt};
        \addlegendentry{single cell}
    
    \addplot[
        color=red,
        mark=triangle,
        ]
        table[x = R, y = Pe]{task3_series.txt};
        \addlegendentry{2 in series}
    
    \addplot[
        color=green,
        mark=*,
        ]
        table[x = R, y = Pe]{task3_parallel.txt};
        \addlegendentry{2 in parallel}
    \end{axis}
\end{tikzpicture}

\section*{Conclusion}
\raggedright
Based on theese curves, we can say that:
\begin{itemize}
\item For in parallel configuration higher current is observed in compare with single cell and in series configurations
\item For multiple sollar cells connected in parallel or in series higher electrical power is observed (it is simply becuase there are more cells)
\item Hovewer, maximum efficiency is the same in all cases
\item Maximum value of efficiency is hited when voltage isn't very law and isn't very high
\item Maximum efficiency in the case of a series connection is achieved at higher voltage values than in other cases
\end{itemize}

\end{document}
